\documentclass{article}
% Commands
\newcommand{\ASSNMT}{Project Necromancer}
\newcommand{\CLASS}{Verification and Validation Checklist}
\newcommand{\Footer}{Grand Valley State University}

\newcommand{\DATE}{July 2015}

% Packages
\usepackage[utf8]{inputenc}
\usepackage[T1]{fontenc}
\usepackage{lmodern}
\usepackage{pdflscape}
\usepackage{geometry}
\usepackage[usenames,dvipsnames]{xcolor}
\usepackage{graphicx}
\usepackage{mathtools}
\usepackage[justification=centering]{caption}
\usepackage{amssymb}
\usepackage[pdftex, pdfborderstyle={/S/U/W 0}]{hyperref} % this disables the boxes around links]
\usepackage{float}
\usepackage{listings}
% \usepackage{color}
\usepackage{enumitem}
\usepackage{fancyhdr}
\usepackage{caption}
\numberwithin{figure}{section}
\usepackage{amsmath}

\numberwithin{equation}{section}
% lstlisting
\definecolor{dkgreen}{rgb}{0,0.6,0}
\definecolor{gray}{rgb}{0.5,0.5,0.5}
\definecolor{mauve}{rgb}{0.58,0,0.82}
\lstset
{
  frame=single,
  frameround=tttt,
  language=C,
  numberstyle=\tiny\color{gray},
  keywordstyle=\color{blue},
  commentstyle=\color{dkgreen},
  stringstyle=\color{mauve},
  tabsize=3,
  breaklines=true,
  basicstyle={\small\ttfamily},
  xleftmargin=\fboxsep,
  xrightmargin=-\fboxsep,
  numbers = left,
  stepnumber = 5,
  firstnumber = 1
}

% macro for appendix to be printed as "Appendix A {name of appendix}"
% instead of "A {name of appendix}"
% From: http://tex.stackexchange.com/questions/160839/having-appendix-a-instead-of-a-appendix
\makeatletter
%% The "\@seccntformat" command is an auxiliary command
%% (see pp. 26f. of 'The LaTeX Companion,' 2nd. ed.)
\def\@seccntformat#1{\@ifundefined{#1@cntformat}%
   {\csname the#1\endcsname\quad}  % default
   {\csname #1@cntformat\endcsname}% enable individual control
}
\let\oldappendix\appendix %% save current definition of \appendix
\renewcommand\appendix{%
    \oldappendix
    \newcommand{\section@cntformat}{\appendixname~\thesection\quad}
}
\makeatother
% Sign and Date command
\newcommand{\namesigdate}[2][5cm]{%
  \begin{tabular}{@{}p{#1}@{}}
    #2 \\[2\normalbaselineskip] \hrule \\[0pt]
    {\small \textit{Signature}} \\[2\normalbaselineskip] \hrule \\[0pt]
    {\small \textit{Date}}
  \end{tabular}
}


\begin{document}
% =====----- Initial Set Up -----=====
% Title Page
\newgeometry{top=2cm,left=1cm,bottom=1cm,right=1cm}
\begin{flushleft}
\pagenumbering{gobble}

\textsc{\LARGE \bfseries \ASSNMT}\\

\textsc{\Large \CLASS}\\[0.2cm]
\linethickness{0.5mm}
{\color{ForestGreen}\line(1,0){350}} \\ [1.0cm]

\begin{flushleft} \large
\begin{tabular}{lll}
  Sponsored By: & The Rainforest Connection (RFCx) \includegraphics[height=0.4cm]{rfcxlogo} & \\
                &               & \\
  Submitted By: & Joe Gibson    & \href{mailto:gibsjose@mail.gvsu.edu}{gibsjose@mail.gvsu.edu}\\
              & David Adlof     & \href{mailto:adlofd@mail.gvsu.edu}{adlofd@mail.gvsu.edu}\\
              & Kalee Stutzman  & \href{mailto:stutzmak@mail.gvsu.edu}{stutzmak@mail.gvsu.edu}\\
              & Jesse Millwood  & \href{mailto:millwooj@mail.gvsu.edu}{millwooj@mail.gvsu.edu}\\
\end{tabular}

\bigskip

\bigskip
Date Submitted: \DATE
\end{flushleft}

\smallskip
{\color{ForestGreen}\line(1,0){350}} \\ [1.0cm]

% Fill in the rest of the cover page
\vfill

% Bottom of the page
\begin{center}
{\large \Footer}
\end{center}
\begin{figure}[H]
  \centering
  \includegraphics[width=.1\textwidth]{small_gvsu}
\end{figure}
\end{flushleft}
\restoregeometry
\newpage
% Define Page Geometry for rest of report
{\newgeometry{left=0.8in, right=0.8in, top=1in, bottom=1in}
% Page Numbers
\pagenumbering{arabic}
\pagestyle{fancy}
\fancyhf{}
\lhead{\ASSNMT}
\rhead{\leftmark}
\rfoot{Page \thepage}
% No paragraph indents
\setlength{\parindent}{0cm}
% =====----- Rest of Report -----=====
\newpage
\tableofcontents
\newpage
%%%%%%%%%%%%%%%%%%%%%%%%%%%%%%%%%%%%%%%%%%%%%%%%%%%%%%%%% DESCRIPTION %%%%%%%%%%%%%%%%%%%%%%%%%%%%%%%%%%%%%%%%%%%%%%%%%%%%%%%%%%
\section{Description} \label{sect:description}
This document serves as a checklist for both verification and validation of the RFCx Project Necromancer software and hardware specifications as outlined in the Design Document. Wherein any specifications changed or were removed, this document also describes the modifications.

\newpage

\section{Verification} \label{sect:verification}
This section outlines the steps that are planned to verify that the hardware and software operate in the intended manner.
\subsection{Hardware Verification} \label{sect:hardwareverification}
To verify the power consumption of the devices in \ref{HWa}. the voltage applied and current drawn by the both the enhanced device and the current device will be measured over the course of 1 hour during a realistic usage scenario. The average power will be verified to not exceed 90\% of the existing device’s usage.

To verify \ref{HWf}. a visual inspection by volunteers will qualify camouflage schemes. A successful inspection will be one in which more than 80\% of the volunteers are unable to find the device within a 2 minute inspection window.

To verify \ref{HWg}. a sample group of people with varying skill levels will be asked to provide feedback the on assembly process.

\subsection{Software Verification} \label{sect:softwareverification}
To verify \ref{SWa}., audio will be recorded and compressed using multiple algorithms including the current algorithm. An infographic hosted on RFCx’s website, shown in Figure 2 illustrates the flow of data that is collected from the phone and analyzed on the server.

To verify \ref{HW3}, simple packets will be sent from the microcontroller to the phone. These packets can be verified with debug tools or simple programs on the microcontroller and the phone. A debug serial port may be included as a peripheral to the microcontroller to aid in debugging. Any analog values read by the microcontroller will be verified with shop equipment.

\newpage

\section{Validation} \label{sect:validation}
\ref{HW3}, \ref{HWb}, and \ref{HWd} can be validated by inspecting the components on the PCB to ensure all required components are present on the board. Requirements \ref{HWg} and \ref{HWh} can be validated by handing the assembly instructions out to a sample group of people with varying skill levels and receiving feedback on the set of instructions from them. People with the ability to read other languages can be used if the documents have been translated from English into other languages.

\ref{HW4} can be validated by inspecting the total cost printed on the BOM for the enhanced device.

\ref{HW1}, \ref{HWa}, and \ref{HW3} can be validated by current and voltage measurements taken in the lab using a DMM. The power calculations made based on those measurements can be compared to similar calculations based on measurements taken from the current device. SPICE simulation data and values displayed on the RFCx Sentinel app will be compared to the actual power consumption of the new device.

\ref{SW1}, \ref{SW2}, \ref{SW3}, and \ref{HW3} can be validated by comparing the values reported in the RFCx Sentinel app to the calculated power consumption of the enhanced device.

\ref{HWc} can be validated by changing the scale factor in the microcontroller program, reprogramming the microcontroller from the PCB header and comparing the values displayed on the RFCx Sentinel app to the previous displayed values. The new values displayed on the RFCx Sentinel app should reflect the change to the scale factor in the microcontroller program.

\ref{HWc}, \ref{SW3}, and \ref{SWa} can be validated during a field test. The enhance device can be mounted at eye level in a tree and turned on to allow the device to start sending audio data to the RFCx server. Requirement \ref{HWf} can be validated by having a volunteer outside of the project search for the device in a wooded area without knowing what it looks like to see if it sticks out. A chainsaw can be turned on 0.25 miles from the device to trigger an alert and validate that the enhanced device can communicate with the RFCx server. The size of the audio recording sent to the RFCx server after employing the audio compression algorithm can be compared to the size of audio data sent to the RFCx server by the current device to determine the audio compression ratio of this algorithm. Both audio recordings can be played back to determine the level of audio interference created by GSM transmission.

%%%%%%%%%%%%%%%%%%%%%%%%%%%%%%%%%%%%%%%%%%%%%%%%%%%%%%%%% VERIFICATION %%%%%%%%%%%%%%%%%%%%%%%%%%%%%%%%%%%%%%%%%%%%%%%%%%%%%%%%%
\section{Verification Checklist} \label{sect:verification_chk}
Verification description...

\begin{enumerate}[align=left,leftmargin=*, labelindent= 0em, label=\textbf{Requirement \thesubsubsection.\arabic*.}, itemindent=0em]
\item \label{SW1}The microcontroller software should be capable of bi-directional communication between itself and the Android phone.
\item \label{SW2}The microcontroller software should be capable of communicating power usage diagnostics to the phone over the ADB protocol.
\item \label{SW3}An Android companion application may be capable of reporting power diagnostics from the microcontroller for testing and debugging.
\item \label{SW4}There shall be a header on the PCB to program the microcontroller.
\end{enumerate}

\newpage

%%%%%%%%%%%%%%%%%%%%%%%%%%%%%%%%%%%%%%%%%%%%%%%%%%%%%%%%% VALIDATION %%%%%%%%%%%%%%%%%%%%%%%%%%%%%%%%%%%%%%%%%%%%%%%%%%%%%%%%%%%
\section{Validation Checklist} \label{sect:validation_chk}
Validation description...

\bigskip
The following is a list of the main ICs that comprise the PCB and are outlined in this section:
\begin{description}[font=$\bullet$\scshape\bfseries]
\item[Atmel ATMega328P]: 8-bit AVR microcontroller (Appendix \ref{fig:atmeldat})
\item[SPV1040 Max Point Power Tracker]: Solar power controller and lithium-ion battery charger (Appendix \ref{fig:spvdat})
\item[ADS1015]: External 4-input 12-bit Analog-to-Digital Converter, $I^2 C$ communication (Appendix \ref{fig:adsdat})
\item[LM75BD]: Ultra Low Power Temperature Sensor,$\pm 2^{\circ}C$, $I^2 C$ communication (Appendix \ref{fig:lm75dat})
\item[(Optional) HIH6130]: 14-bit resolution Humidity and Temperature Sensor with accuracy of $\pm5$\% Relative Humidity and $\pm 1^{\circ}C$, $I^2 C$ communication (Appendix \ref{fig:hihdat})
\item[FT230X]: FTDI USB-UART interface for USB serial communication (Appendix \ref{fig:ftdidat})
\item[BQ2057CTS]: Linear Battery Charge Management IC (Appendix \ref{fig:bq2057dat})
\item[LM61428]: Simple Switcher Boost Controller IC (Appendix \ref{fig:lmrdat})
\item[SM72238]: Micropower Fixed 3.3V LDO (Appendix \ref{fig:sm72dat})
\item[LTC6800]: Rail-To-Rail Input and Output Instrumentation Amplifier (Appendix \ref{fig:ltc6800dat})
\item[LTC4412]: Low loss powerpath controller (Appendix \ref{fig:ltc4412dat})
\end{description}

\newpage

\section{Sponsor Sign-off} \label{sect:signoff}
On behalf of Rainforest Connection (RFCx), the undersigned approves the checklist of verification and validation items and the functional requirements and design contained in the design document that specify the project deliverables and functionality.

\bigskip

\bigskip
\noindent \namesigdate{Topher White} \hfill \namesigdate{Dave Grenell}

\end{document}

% ======== For Reference =============
% H parameter for the figure environment
% keeps it from floating
\begin{figure}[H]
	\centering
	\includegraphics[width=0.8\textwidth]{FIG2_1b}
	\caption{Convolution of Identical Unit Step Sequences}
	\label{fig:21b}
\end{figure}
% To turn off caption numbering place this:
%\captionsetup[figure]{labelformat=empty}
% Before the figure, To turn back on:
%captionsetup[figure]{labelformat=default}
\begin{equation}
	\begin{array}{rcl}

	y[n] &=& 0.5x[n]+x[n-1]+2x[n-2]\\
	y[n] &=& 0.8y[n-1]+2x[n]\\
	y[n]-0.8y[n-1] &=& 2x[n-1]
	\end{array}
\end{equation}

\begin{subequations}
	\begin{align}
		h[n] &= 2\delta[n+1]-2\delta[n-1]\\
		x[n] &= \delta[n] + \delta[n-2]
	\end{align}
\end{subequations}

% Resize used to make table width of text, may be omitted
\begin{table}[h]
\resizebox{\textwidth}{!}{
\centering
\begin{tabular}{|c|c|c|c|}
\hline
Case & $Z_c$          & $l$           & $c$       \\ \hline
1    & $110.9 \Omega$ & 0.593 $\mu H$ & 0.048 nF  \\ \hline
2    & $171.3 \Omega$ & 0.803 $\mu H$  & 27.408 pF \\ \hline
3    & $327.3 \Omega$ & 1.102 $\mu H$  & 10.294 pF \\ \hline
\end{tabular}}
\caption{Calculated Parameters}
\label{tbl:calcd}
\end{table}

% Code Snippet:
\begin{lstlisting}[language=C,label=lala,caption=this thing]
  code snippet
\end{lstlisting}

% bulleted list
\begin{itemize}
\item this is an item
\end{itemize}

\renewcommand*\contentsname{ }
\tableofcontents
\listoffigures
